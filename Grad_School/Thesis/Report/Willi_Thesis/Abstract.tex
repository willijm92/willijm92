%Abstract Page 

\hbox{\ }

\renewcommand{\baselinestretch}{1}
\small \normalsize

\begin{center}
\large{{ABSTRACT}} 

\vspace{3em} 

\end{center}
\hspace{-.15in}
\begin{tabular}{ll}
Title of thesis:		& {\large  ANALYSIS OF PROPANE GAS FIRE}	\\
							& {\large  EXPERIMENTS AND SIMULATIONS} 	\\
							& {\large  IN RESIDENTIAL SCALE STRUCTURES} \\
\ \\
							& {\large  Joseph M. Willi, Master of Science, 2017} \\
\ \\
Dissertation directed by: 	& {\large  Professor James A. Milke} \\
							& {\large  Department of Fire Protection Engineering} \\
\end{tabular}

\vspace{3em}

\renewcommand{\baselinestretch}{2}
\large \normalsize

Nine full-scale fire experiments were conducted in two residential-scale structures with a fire source provided by three propane gas burners. Five of the experiments were conducted in a single-story structure, and four were conducted in a two-story structure. The structures were instrumented to measure temperature; oxygen and carbon dioxide gas concentrations; gas velocity; and heat flux. Various doors and vents were opened and closed during the experiments to provide passive horizontal and vertical ventilation. Numerical simulations of the nine experiments were conducted using Fire Dynamics Simulator (FDS) (version 6.5.3). The simulation results were compared to the experimental data, and overall, the predicted temperatures, gas concentrations, and heat fluxes from the simulations were within the expected agreement range based on the calculated experimental and model relative standard deviations and model bias factors provided by the FDS Validation Guide for each quantity. The one significant discrepancy between expected agreement and actual agreement of the simulation data and experimental data occurred with the gas velocity measurements, which produced a model relative standard deviation that was 0.18 larger than the value provided in the FDS Validation Guide. Overall, comparing the experimental results to the FDS simulation results suggest that there is sufficient agreement between the two and that FDS is capable of accurately modeling certain fire scenarios in one and two story, residential-sized structures.

% However, if the average gas velocity across the entire vent was to be considered instead of individual point measurements, the uncertainty is only 0.05 greater than the values presented in the guide.

