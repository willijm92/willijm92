%Abstract Page 

\hbox{\ }

\renewcommand{\baselinestretch}{1}
\small \normalsize

\begin{center}
\large{{ABSTRACT}} 

\vspace{3em} 

\end{center}
\hspace{-.15in}
\begin{tabular}{ll}
Title of thesis:		& {\large  ANALYSIS OF PROPANE GAS FIRE}	\\
							& {\large  EXPERIMENTS AND SIMULATIONS} 	\\
							& {\large  IN RESIDENTIAL SCALE STRUCTURES} \\
\ \\
							& {\large  Joseph M. Willi, Master of Science, 2017} \\
\ \\
Thesis directed by: 	& {\large  Professor James A. Milke} \\
							& {\large  Department of Fire Protection Engineering} \\
\end{tabular}

\vspace{3em}

\renewcommand{\baselinestretch}{2}
\large \normalsize

Nine full-scale fire experiments were conducted in two residential-sized structures with a fire source provided by three propane gas burners. Five of the experiments were conducted in a single-story structure, and four were conducted in a two-story structure. The structures were instrumented to measure temperature; oxygen and carbon dioxide gas concentrations; gas velocity; and heat flux. Various doors and vents were opened and closed during the experiments to change the ventilation through the structures. Numerical simulations of the nine experiments were conducted using Fire Dynamics Simulator (FDS) (version 6.5.3). The model data  were compared to the corresponding experimental data, and the temperature, gas species concentration, and heat flux data produced by the simulations were within the expected agreement range based on the values of experimental relative standard deviation, model relative standard deviation, and model bias factor provided by the FDS Validation Guide for each specific data type. The one significant discrepancy between the simulation data and experimental data occurred with the gas velocity measurements, which produced a model relative standard deviation that was 0.18 larger than the value from the FDS Validation Guide. Overall, comparing the FDS simulation output to the experimental data shows sufficient agreement between the predicted and measured data, thus indicating that FDS is capable of accurately modeling different aspects of fire scenarios in residential-sized structures.

% However, if the average gas velocity across the entire vent was to be considered instead of individual point measurements, the uncertainty is only 0.05 greater than the values presented in the guide.

