%%%%%%%%%%%%%%%%%%%%%%%%%%%
% CHAPTER 1: Introduction %
%%%%%%%%%%%%%%%%%%%%%%%%%%%

\renewcommand{\thechapter}{1}

\chapter{Introduction}
The development and behavior of compartment fires, such as those inside residential-scale structures, depend greatly on the ventilation conditions within the compartment. Nine full-scale fire experiments were conducted to study how opening and closing different doors and other vents affect ventilation and the fire environment within residential-sized structures. The experiments were conducted in two experimental structures designed to replicate a single-story and a two-story dwelling. The fire source for each experiment was provided by three gas propane burners. The flow of propane to the burners was controlled by a high-precision turn valve and the total displaced gas volume was measured using a rotary gas meter. Local measurements of temperature, gas concentrations, gas velocity, and heat flux were collected at various locations throughout the structure while the ventilation within the structure was varied. During a number of the tests, a PPV fan was used in conjunction with the opening and closing of vents to further affect the ventilation conditions within the fire environment.

Numerical simulations of the nine tests were performed using the program Fire Dynamics Simulator (FDS) (version 6.5.3)~\cite{FDS_Users_Guide}. FDS is a computational fluid dynamics (CFD) model of thermally-driven fluid flow that is developed and maintained by the National Institute of Standards and Technology (NIST). FDS numerically solves a form of the Navier-Stokes equations for low-speed ($Ma < 0.3$), thermally-driven flows with an emphasis on smoke and heat transport from fires. The FDS Technical Reference Guide~\cite{FDS_Tech_Guide} provides a complete description of the model, including the formulation of the equations and numerical algorithm utilized by the software. FDS is mathematically verified~\cite{FDS_Verification_Guide} and validated against a continually growing database of experimental data from a variety of fire scenarios~\cite{FDS_Validation_Guide}.

Fire protection engineers commonly use CFD models to predict fire dynamics and smoke movement for potential fire scenarios as they are developing certain fire safety designs. FDS is the most commonly used program for this type of application. Therefore, it's crucial that the program is validated for a wide range of fire scenarios, including those within residential-scale structures. Currently, there are a limited number of experiments for which FDS has been validated that involve fires within residential-scale structures. Furthermore, there are no cases described within the FDS Validation Guide that involve multi-story residential-scale structures. So, generating FDS simulations of the nine gas burner experiments and comparing the model data to the experimental data from the burner tests will provide a crucial addition to the FDS validation database.

This report contains a thorough description of the experimental structures and instrumentation used to collect data during the nine propane burner experiments. The procedures followed during each test are also outlined in detail. Following the experimental descriptions, the methodology used to create and execute the FDS simulations of each gas burner experiment is presented. Following the model description, the data output by the models are compared to the corresponding sensor measurements of temperature; $O_2$ and $CO_2$ gas concentration; gas velocity; and heat flux. Plots of simulation data and experimental data over the duration of the tests are presented in addition to log/log scatter plots of the predicted data versus the corresponding sensor data. Relative standard deviation values for the model and experimental data were calculated and used to determine a model bias factor for each data type. Finally, the standard deviation values and model bias factors are compared to the values listed within the FDS Validation Guide and any discrepancies between the two are discussed. 