%%%%%%%%%%%%%%%%%%%%%%%%%%%
% CHAPTER 1: Introduction %
%%%%%%%%%%%%%%%%%%%%%%%%%%%

\renewcommand{\thechapter}{1}

\chapter{Introduction}
The development and behavior of compartment fires, such as those inside residential structures, depend greatly on the ventilation conditions within the compartment. Nine full-scale fire experiments were conducted in residential-sized structures to study how opening and closing different doors and vents affect ventilation and the fire environment. Two experimental structures designed to replicate a single-story and a two-story dwelling were used. The fire source for each experiment was provided by three gas propane burners. The flow of propane to the burners was controlled by a high-precision turn valve and the total displaced gas volume was measured using a rotary gas meter. Local measurements of temperature, gas species concentration, gas velocity, and total heat flux at various locations throughout the structure were collected while the ventilation within the structure was varied through the opening and closing of doors and vents. 

% During a number of the tests, a positive pressure ventilation (PPV) fan was used at the end.

Numerical simulations of the nine tests were performed using the program Fire Dynamics Simulator (FDS) (version 6.5.3)~\cite{FDS_Users_Guide}. FDS is a computational fluid dynamics (CFD) model of thermally-driven fluid flow that is developed and maintained by the National Institute of Standards and Technology (NIST). FDS numerically solves a form of the Navier-Stokes equations for low-speed ($Ma < 0.3$), thermally-driven flows with an emphasis on smoke and heat transport from fires. The FDS Technical Reference Guide~\cite{FDS_Tech_Guide} provides a complete description of the model, including the formulation of the equations and numerical algorithm utilized by the software. FDS is mathematically verified~\cite{FDS_Verification_Guide} and validated against a continually growing database of experimental data from a variety of fire scenarios~\cite{FDS_Validation_Guide}.

Fire protection engineers commonly use CFD models to predict fire dynamics and smoke movement for potential fire scenarios as they are developing certain fire safety designs. FDS is the most commonly used program for this type of application. Therefore, it's crucial that the program is validated for a range of fire scenarios, including those within residential structures. Currently, there are a limited number of experiments for which FDS has been validated that involve fires within residential-sized structures. Furthermore, there are no cases described within the FDS Validation Guide that involve fire scenarios inside multi-story residential scale structures. Thus, generating FDS simulations of the nine gas burner experiments and comparing the results to the experimental data from the nine gas burner tests will provide an important addition to the FDS validation database.

This report contains a thorough description of the experimental structures and instrumentation used to collect data during the nine propane gas fire experiments. The procedures followed during each test are also outlined. Following the description of the experimental setup and procedures, the FDS input files that define experiment simulations are discussed in detail. Next, the data output by the models are compared to the corresponding sensor data of temperature; oxygen and carbon dioxide concentration; gas velocity; and heat flux. Figures of simulation data and experimental data plotted over the duration of the tests are presented alongside log/log scatter plots that summarize the overall results for each data type. The relative standard deviation values for the model and experimental data and the resulting model bias factor are reported with each summary plot. Then, the relative standard deviation values and model bias factor for each data type are compared to the corresponding values listed within the FDS Validation Guide. 