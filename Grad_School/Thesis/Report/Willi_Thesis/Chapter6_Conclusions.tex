%%%%%%%%%%%%%%%%%%%%%%%%%
% CHAPTER 6: Conclusion %
%%%%%%%%%%%%%%%%%%%%%%%%%

\renewcommand{\thechapter}{6}

\chapter{Conclusion}

Nine full-scale fire tests were conducted in two residential-sized structures. Five of the experiments occurred in a single-story structure with three different rooms, and the other four experiments were performed in a two-story structure with a ground level having an open floor plan and the second level having two rooms and a hallway. The fire source for each experiment was provided by a set of three diffusion flame burners with propane as the fuel. Various doors and vents were opened and closed during each test to change ventilation within the structure. Local measurements of temperature, gas velocity, heat flux, and gas concentrations were collected at various locations throughout the structure during the experiments. 

The dimensions of each structure were carefully measured, and their construction materials were well-defined. The locations of the experimental instrumentation were also measured and the times of different experimental events were recorded. Additionally, the total volume of propane delivered to the burners was measured by a rotary gas meter and was used to calculate the heat release rate of the fire during the various tests. Using this information as input data, simulations of the experiments were created and executed using NIST's Fire Dynamics Simulator --- the most common CFD modeling software used by fire protection engineers to predict fire dynamics and smoke movement for potential fire scenarios. The simulation results were compared to the experimental data from the experiments.

The agreement between the FDS simulation data and experimental data for the gas burner experiments is consistent with the statistical values given by the FDS Validation Guide. For the quantities of HGL temperature, ceiling jet temperature, and total heat flux, the model bias value was equal to or better than (closer to the ideal value of 1) the overall model bias values stated within the FDS Validation Guide. Similarly, the relative standard deviations of the experimental data and model data were equal to or less than (more accurate) the values the same parameters provided by the validation guide for each of the three data quantities. The model bias and experimental and model relative standard deviations calculated for the $O_2$ and $CO_2$ gas concentration data from the gas burner tests were very close to or better than the values documented in the FDS Validation Guide. The most significant discrepancy between the values calculated from the gas burner test data and those documented in the FDS Validation Guide was associated with the gas velocity data comparison. The difference could be a result of the fact that the tests were conducted outdoors, that the instrumentation used to measure gas velocity had a relatively large uncertainty range, and/or the limitations of LES CFD models to effectively model turbulent gas flow at fine resolutions.

Overall, the comparison of the simulation data to the experimental data suggests that the accuracy of the FDS models of the gas burner experiments in residential-scale structures is sufficient and comparable to the accuracy of the other FDS models included in the FDS Validation Guide.
