\documentclass{article}
\usepackage{outlines}
\usepackage{enumitem}
\setenumerate[1]{label=\Roman*.}
\setenumerate[2]{label=\Alph*.}
\setenumerate[3]{label=\roman*.}
\setenumerate[4]{label=\alph*.}

\begin{document}

Using Python/Anaconda in Fire Research \\

\begin{outline}[enumerate]
\1 Introduction 
	\2 Type of work done in Fire Research Division at NIST [kickass fire picture]
	\2 Areas in which Python/Anaconda is used
		\3 Realtime plotting helmet data
		\3 FF LODI/LODD interactive map
		\3 Realtime plotting of data from FDS model rendering

\1 Interactive map (CGW)

\1 Portable Measurement and Data Acquisition System
	\2 Overview of system [pic of entire system together]
		\3 Helmet portion [pic of helmet]
			\4 Thermocouple 
			\4 Heat flux gauge
			\4 Cooling water lines
		\3 Pack portion	[pic of pack]
			\4 Miniature pump
			\4 Water reservoir
			\4 Arduino Yun as data logger
	\2 Setting up Arduino Yun and host computer 
		\3 Arduino Yun
			\4 \textit{Details about all the packages and other shit to set up Yun?}
			\4 SD Card
		\3 Host Computer
			\4 RabbitMQ message broker server
			\4 Run receive\_helmet\_data
			\4 Deploy Bokeh server
			\4 Run plot\_helmet\_data.py 
	\2 Plotting heat flux and temperature data in real time [fig of previous workflow and new work flow using arduino/bokeh]
		\3 Arduino Yun
			\4 Adafruit + arduino code to receive voltages from sensors
			\4 send\_helmet\_data.py executed using Yun's Linux distribution (OpenWrt-Yun) and following tasks are performed: \\ 
			\begin{itemize}
				\item Sensor voltages converted to significant measurement (temperature or heat flux value) 
				\item Arduino Yun connects to message broker on host computer using Yun's built-in WiFi support, IP address specified by user, and Pika package
				\item Message containing data at current time-step is constructed and published to message broker on host computer and locally to Yun's SD card \\
			\end{itemize}
		\3 Host Computer
			\4 receive\_helmet\_data.py connects to the message broker, receives message sent to the broker from the Yun, prints message in terminal, and writes data to .csv file on host computer
			\4 plot\_helmet\_data.py reads .csv file and plots corresponding data every second; new line of data added to .csv file every second

\1 Summary

\end{outline}

\end{document}